% Template for Cogsci submission with R Markdown

% Stuff changed from original Markdown PLOS Template
\documentclass[10pt, letterpaper]{article}

\usepackage{cogsci}
\usepackage{pslatex}
\usepackage{float}
\usepackage{caption}

% amsmath package, useful for mathematical formulas
\usepackage{amsmath}

% amssymb package, useful for mathematical symbols
\usepackage{amssymb}

% hyperref package, useful for hyperlinks
\usepackage{hyperref}

% graphicx package, useful for including eps and pdf graphics
% include graphics with the command \includegraphics
\usepackage{graphicx}

% Sweave(-like)
\usepackage{fancyvrb}
\DefineVerbatimEnvironment{Sinput}{Verbatim}{fontshape=sl}
\DefineVerbatimEnvironment{Soutput}{Verbatim}{}
\DefineVerbatimEnvironment{Scode}{Verbatim}{fontshape=sl}
\newenvironment{Schunk}{}{}
\DefineVerbatimEnvironment{Code}{Verbatim}{}
\DefineVerbatimEnvironment{CodeInput}{Verbatim}{fontshape=sl}
\DefineVerbatimEnvironment{CodeOutput}{Verbatim}{}
\newenvironment{CodeChunk}{}{}

% cite package, to clean up citations in the main text. Do not remove.
\usepackage{apacite}

% KM added 1/4/18 to allow control of blind submission


\usepackage{color}

% Use doublespacing - comment out for single spacing
%\usepackage{setspace}
%\doublespacing


% % Text layout
% \topmargin 0.0cm
% \oddsidemargin 0.5cm
% \evensidemargin 0.5cm
% \textwidth 16cm
% \textheight 21cm

\title{The interactions of rational, pragmatic agents\\
lead to efficient language structure and use}


\author{{\large \bf Benjamin N. Peloquin} \\ \texttt{bpeloqui@stanford.edu} \\ Department of Psychology \\ Stanford University \And {\large \bf Noah D. Goodman} \\ \texttt{ngoodman@stanford.edu} \\ Department of Computer Science \\ Stanford University \And {\large \bf Michael C. Frank} \\ \texttt{mcfrank@university.edu} \\ Department of Psychology \\ Stanford University}

\begin{document}

\maketitle

\begin{abstract}
Languages display a diverse set of distributional regularities such as
the relation between a word's frequency and rank in a corpus, the
distribution of dependency lengths, or the presence of lexical
properties such as ambiguity. We discuss a framework for studying how
these properties emerge from in-the-moment interactions of rational,
pragmatic speakers and listeners. Our work takes notions of
lexicon-level efficiency as a starting point, connecting these ideas to
notions of conversational-level efficiency. To do so, we derive an
objective function for measuring the communicative efficiency of
linguistic systems and then examining the behavior of this objective in
a series of simulations focusing on the communicative function of
ambiguity in language. These simulations suggest that rational pragmatic
agents will produce communicatively efficient systems and that
interactions between such agents provide a framework for examining
efficient properties of language more broadly.

\textbf{Keywords:}
Communicative efficiency, Rational Speech Act theory, computational
modeling, information theory, agent-based simulation
\end{abstract}

\section{Introduction}\label{introduction}

Why do languages look the way they do? Zipf (1949) proposed that
distributional properties found in natural language were evidence of
speaker-listener effort minimization. In his own words, ``we are arguing
that people do in fact act with a maximum economy of effort, and that
therefore in the process of speaking-listening they will automatically
minimize the expenditure of effort.'' Evidence for this claim has been
largely derived at the level of the lexicon. Zipf argued that the
particular relationship between a word's frequency and its rank, length,
and denotation size could be explained as an emergent property of
speaker-listener effort minimization. \par

Zipf articulated what is now considered a \emph{functionalist} approach
to language science -- analyzing language structure and use in terms of
efficiency. Such an approach might reframe our opening question as
follows -- how does having \textbf{property x} make using language
\(\ell\) more or less useful for communication? This reframing provides
an opportunity to study both language \emph{structure} (as Zipf
primarily did) or \emph{use} in terms of efficiency. For example, Regier
et al. (2015) showed that languages appear to organize semantic domains
(form-meaning mappings) in maximally efficient ways. Likewise,
Piantadosi et al. (2011) argued that lexical ambiguity is an efficient
property of any communication system when communication is
contextualized.\par

Zipf's original work and subsequent functionalist projects describe a
varied set of linguistic properties and human behaviors. Common to many
is a characterization of a fundamental effort-asymmetry underlying
everyday communication. Simply put, what is ``hard'' for a speaker is
likely different from what is ``hard'' for a listener. Zipf
characterized this as follows -- purely from the standpoint of speaker
effort, an optimal language \(\ell_{speaker}^*\) would tend toward a
vocabulary of just a single, low-cost word. Given such a language, the
full set of potential meanings would be conveyed using only that word,
i.e. \(\ell_{speaker}^*\) would be fully ambiguous and all possible
meanings would need to be disambiguated by a listener. From the
standpoint of listener effort, an optimal language \(\ell_{listener}^*\)
would map all possible meanings to distinct words, removing a listener's
need to disambiguate. In this example, speaker effort is related to
\emph{production cost} and listener effort to \emph{understanding or
disambiguation cost}. Clearly natural languages fall between the two
extremes of \(\ell_{speaker}^*\) and \(\ell_{listener}^*\). Zipf
proposed that the particular lexicon-level properties he observed were a
result of these competing forces -- the pressure to jointly minimize
speaker and listener effort.\par

But how does such optimization take place? The example given by Zipf
(1949), describes local, communicative interactions in terms of a
\textit{reference game}. Speakers intend to refer to some object in the
world \(m\). They choose some utterance \(u\) to transmit this intended
meaning, \(u \rightarrow m\). The listener attempts to reconstruct this
intended meaning given the transmitted utterance, \(m \rightarrow u\).
Other functionalist projects have assumed this basic reference game
setting (Piantadosi et al., 2011; Regier et al., 2015) and this
simplification of the communicative act has proven productive in
theoretical (Ferrer-i-Cancho, 2018), simulation-based (Kirby, Griffiths,
\& Smith, 2014) and empirical explorations (Hawkins, Franke, Smith, \&
Goodman, 2018) of efficient language structure and use.\par

Zipf's conception of speaker and listener effort may be connected to
speakers' pragmatic reasoning in the moment. Under a Gricean treatment
of pragmatics, speakers and listeners are assumed to follow a set of
conversational maxims in which they cooperate to transfer information
(Grice, 1975). These maxims appear to emerge from efficiency concerns,
however (Horn, 1984). A primary target of the current work is to
formalize this connection, by providing a framework for deriving
system-level efficiencies from local, interaction behavior. Our claim is
that to understand an ``efficient'' property of a system we need to
consider \emph{how} that property is \emph{used} efficiently.
Additionally, any quantitative description of ``efficiency'' should be a
function of both the system and users.\par

In the current work we provide a case-study for the kind of framing
presented above. We choose a property of languages that could, in
principle, vary freely, but shows strong regularities across languages.
The explanadum is why this regularity holds. We examine lexical
ambiguity as our property, extending ideas by Piantadosi et al. (2011).
We define a novel measure of efficiency that depends on the
interactional behavior of interlocutors. For our model of speakers and
listeners we adopt the Rational Speech Act (RSA) framework, which is
nicely attested in experimental data on pragmatic interaction. With
these basic ingredients we have a measure of efficiency given language
and speaker-listener agents. Finally, we show that the property of
interest is prevalent in languages that optimize this measure of
efficiency and also how this property is used efficiently during local
interaction. The primary contribution of this work is to show how the
reference game frameowrk, in combination with formal models of
communication, can be used to connect ideas about system-level
efficiencies to in-the-moment language use.\par

\section{Exploring efficient language design and use in rational
pragmatic
agents}\label{exploring-efficient-language-design-and-use-in-rational-pragmatic-agents}

\begin{CodeChunk}
\begin{figure}[H]

{\centering \includegraphics{figs/plot-reference-game-1} 

}

\caption[An example reference game with associated literal semantics (in our terminology a ``language'')]{An example reference game with associated literal semantics (in our terminology a ``language'').}\label{fig:plot-reference-game}
\end{figure}
\end{CodeChunk}

\subsubsection{Reference games}\label{reference-games}

Zipf's example of optimal speaker- and listener-languages took the form
of a reference game. We adopt that formulation here, assuming these
communication games as our basic unit of analysis. In this setting,
speakers and listeners are aware of a set of objects \(M\), which we
will refer to as \emph{meanings} and are knowledgeable about the set of
possible signals \(U\) (\emph{utterances}) that can be used to refer to
a given meaning (see Figure 1). Utterances may have different relative
costs, operationalized via a prior over utterances \(P(U)\). Similarly,
meanings differ in the relative degree to which they need to be talked
about, operationalized as a prior over meanings
\(P(M)\)\footnote{The prior over meanings are analogous to the \textit{need probabilities} assumed in previous work Regier, Kemp \& Kay (2015).}.
We consider a set of contexts \(C\) with an associated prior \(P(C)\).
Contexts encode different meaning priors, formalized as separate
conditional distributions over meanings e.g.
\(p(M|C=c_i) \neq p(M|C=c_j)\). Finally, we consider a set of
communicative events \(e \in E\) where \(<u, m> = e\) is a tuple of
utterance, meaning pairs. Note that in the case of single context
\(|C|=1\) we simply have a prior over meanings \(p(M)\), a
simplification we will use in introducing our objective function.\par

\subsubsection{Languages}\label{languages}

A language \(\ell\) defines the set of semantic mappings between
utterance and meanings. For example, Figure 1 contains four utterances
\(U = \{\text{blue}, \text{green}, \text{square}, \text{circle}\} \text{ and three meanings }M = \{\text{green-square}, \text{blue-square}, \text{green-circle}\}\).
The boolean matrix describes the literal semantics of the language
\(\ell\). Note that will consider language to ``contain ambiguous
items'' if there is at least one utterance \(u \in U\) which can apply
to multiple meanings (i.e.
\(|[[u_i]]| > 1\))\footnote{We use double brackets $[[\dots]]$ to represent denotation}.
For example, in Figure 1 both the words ``blue'' and ``green'' are
ambigous so we would say that \(\ell\) contains ambiguity.

\subsubsection{Speakers and listeners}\label{speakers-and-listeners}

The Rational Speech-act framework (RSA) is computational-level theory of
pragmatic language use. RSA can be understood as a formalization of
essential Gricean pragmatic principles -- agents reason about one
another and their shared context. For this reason, we adopt RSA as our
representational framework to operationalize Gricean (rational and
pragmatic) speaker-listeners (Grice, 1975). We refer the reader to
Section 1 of our supplementary materials for a more detailed description
of basic RSA speaker-listener definitions, and to Goodman \& Frank
(2016) for an overview of its appliations.\par

An RSA \emph{speaker agent} defines a conditional distribution over
utterances, mapping from intended meanings \(M\) to utterances \(U\)
using \(\ell\) in a given context \(c\). That is, a speaker defines
\(P_{speaker}(u|m; \ell)\). We will use \(S(u|m; \ell)\) as short-hand
throughout. A \emph{listener agent} defines a conditional distribution
over meanings, mapping from utterances \(U\) to meanings \(M\) using
\(\ell\) in a given context \(c\). We will use \(L(m|u; \ell)\) as
shorthand. Note that both speakers and listeners can induce joint
distributions over the set of all events \(E\), although, importantly,
these distributions may differ:
\[P_{speaker}(u, m; \ell) = S(u|m; \ell)p(m)\]
\[P_{listener}(u, m; \ell) = L(m|u; \ell)p(u)\]

\section{Zipfian objective for linguistic system
efficiency}\label{zipfian-objective-for-linguistic-system-efficiency}

Zipf (1949) proposed that the particular distributional properties found
in natural language emerge from competing speaker and listener
pressures. We operationalize this objective in equation (1) -- the
efficiency of a linguistic system \(\ell\) being used by speaker and
listener agents \(S\) and \(L\) is the sum of the expected speaker and
listener effort to communicate over all possible communicative events
\(E\). Note that to simplify the introduction of this objective we
assume there is just a single context (\(|C|=1\)). In this case
\(p(C=c)=1\) and there is only distribution over meanings
(\(p(m|c) = p(m)\)). We provide a more detailed description of this
derivation in the supplementary materials, both with and without context
\(C\).\par

\begin{equation}
\begin{split}
  \text{Efficiency}(S, L, \ell) = \mathbb{E}_{e \sim P(E)}[\text{speaker effort}] \\+ \mathbb{E}_{e \sim P(E)}[\text{listener effort}]
\end{split}
\end{equation}

We assume that speaker effort is the negative log probability of an
utterance (surprisal) -- intuitively, the number of bits needed to
encode the utterance \(u\). This particular formalization of
speaker-cost is general enough to accommodate a range of instantiations
that might in theory be related to production difficult via articulation
effort, cognitive effort related to lexical access, etc (Bennett \&
Goodman, 2018).\par

\[\text{speaker effort} = -log_2(p(u))\]

We assume listener effort is the negative log probability of the
listener's conditional \(-log_2(L(m|u))\). This operationalization of
listener effort is intuitively related to existing work in sentence
processing in which word difficulty during comprehension is proportional
to surprisal (Hale, 2001; R. Levy, 2008).

\[\text{listener effort} = -log_2(L(m|u; \ell))\]

Importantly, we assume that events \(e = <u, m>\) are sampled according
the to following generative model -- some object occurs in the world
with probability \(p(M=m)\), the speaker attempts to refer to that
object by sampling from her conditional distribution \(S(u|m)\) (i.e.
\(e \sim p(m)S(u|m)\)). From the ingredients highlighted above, it is
possible to derive the following objective between the speaker and
listener joint distributions (see SI 2.2 for a complete derivation).

\begin{equation}
\begin{split}
  = -\mathbb{E}_{P_{speaker}}[log_2(P_{listener})] \\
  = H_{cross}(P_{speaker}, P_{listener}; \ell)
\end{split}
\end{equation}

From an information-theoretic perspective this objective is intuitive:
cross-entropy gives us a measure of dissimilarity between two
distributions -- the average number of bits required to communicate
under one distribution, given that the ``true'' distribution differs. In
our case, this is the difference between the joint distribution assumed
by the speaker \(P_{speaker}\) and listener \(P_{listener}\). In other
words, an ``efficient'' language \(\ell\) minimizes the distance between
what speakers and listeners think.

\section{Simulating the communicative function of
ambiguity}\label{simulating-the-communicative-function-of-ambiguity}

The task of understanding language is marked by a frequent need to
handle various forms of ambiguity; lexical, syntactic, among others
(Wasow, Perfors, \& Beaver, 2005). The ubiquity of this property,
however, has been argued to provide evidence that langauges are not been
optimized for
communication\footnote{Chomsky (2002) famously claimed "If you want to make sure that we never misunderstand one another, for that purpose language is not well designed, because you have such properties as ambiguity." Note that this analysis is analogous to the Zipfian optimal Listener language $\ell_{listener}^*$ we described earlier and define in terms of equation 2 in SI 2.2 .}
(Chomsky, 2002).

Piantadosi et al. (2011) argue just the opposite, claiming that
ambiguity is an \emph{efficient} property of any communication system in
which \emph{communication is contextualized}. Simply put, it is useful
to have a language that re-uses low-cost material (has ambiguity) so
long as the cost of disambiguating the material is low. In particular,
context (or common ground) can provide useful information for
disambiguation.\par

We provide a toy example of their argument here. Say we have two objects
in the world \(m_1\) and \(m_2\), and two languages \(\ell_1\) and
\(\ell_2\). In language \(\ell_1\), the low-cost utterance \(u_1\) can
be used to refer to both \(m_1\) and \(m_2\) (i.e.
\([[u_1]]_{\ell_1} = \{m_1, m_2 \}\) and
\([[u_2]]_{\ell_1} = \emptyset\)). With language \(\ell_2\), however,
the low cost utterance \(u_1\) can only refer to \(m_1\) and a
higher-cost utterance \(u_2\) can refer to \(m_2\) (i.e.
\([[u_1]]_{\ell_2} = \{m_1\}\) and \([[u_2]]_{\ell_2} = \{m_2 \}\)).
While it is cheaper for a speaker to use \(\ell_1\) (because she can
always use a low-cost utterance), it is more difficult for a listener
(because \(u_1\) is ambiguous). If context isn't disambiguating then we
might prefer \(\ell_2\) to \(\ell_1\). But, if context is disambiguating
then the speaker can use \(u_1\) to refer to either \(m_1\) or \(m_2\)
and \(\ell_1\) should be preferred.\par

In the following experiments we explore two aspects of Piantadosi's
claim. First, we examine the efficient language \emph{structure} aspect
of their claim, exploring when the optimal linguistic system \(\ell^*\)
is most likely to contain ambiguous lexical items. In the second
experiment, we explore an efficient language \emph{use} aspect of the
claim -- at what point in a conversation is it useful for a speaker to
use ambiguous lexical material?\par

\begin{CodeChunk}
\begin{figure*}[h]

{\centering \includegraphics{figs/plot-optimal-langs-1} 

}

\caption[Optimal languages are more likely to contain ambiguous items as the amount of contextual information increases]{Optimal languages are more likely to contain ambiguous items as the amount of contextual information increases. Vertical axis shows the proportion of optimal languages that contain ambiguity. Horizontal axis shows the number of contexts in each condition (1-4). Red-line represents the optimal language under our Zipfian cross-entropy objective while the blue and red lines show optimal languages under speaker- and listener-only objectives.}\label{fig:plot-optimal-langs}
\end{figure*}
\end{CodeChunk}

\section{Simulation 1: Optimal languages contain ambiguity when context
is
informative}\label{simulation-1-optimal-languages-contain-ambiguity-when-context-is-informative}

We adopt the framework above, and (following Piantadosi et al. 2011),
add \textit{context} (\(C\)). The authors argue that mapping two
meanings \(m_1\) and \(m_2\) to a single utterance \(u_1\) is useful
when they can be disambiguated \emph{in context}. In our case we
consider a context \(c_i \in C\) to specify a permutation of the meaning
probabilities. That is, \(p(m|c_1) \neq p(m|c_2)\), when there are two
contexts \(|C| = \{c_1, c_2\}\).This in turn, leads to an update to our
linguistic efficiency objective as we now would like to consider the
average speaker-listener effort over all contextualized communicative
events (see section 3 SI).\par

We proceed by generating languages with different amounts of contextual
support (varying the size of \(|C|\)). We search the space of languages,
asessing if the ones which minimize our objective contain ambiguity. If
context leads to more efficient commnication, then optimal languages
should be more likley to be ambiguous as the amount of context
increases.\par

\subsection{Simulation set-up}\label{simulation-set-up}

We conduct \(N=2000\) simulations. For each simulation we enumerate the
set of \emph{valid} languages in which \(|U|=|M|=4\) (recall that \(U\)
is our set of utterances and \(M\) our set of meanings). Note that a
language \(\ell \in L\) is ``valid'' so long as each possible meaning in
\(m \in M\) can be referred to by at least one form \(u \in U\) (every
column of \(\ell\) has some non-zero assignment) and each form maps to
at least one meaning (every column has some non-zero
assignment)\footnote{For the current set of simulations we consider languages in which there are four or five true values -- in a "four" language ambiguity is not possible, but in a "five" language there are ambiguous mappings.}.
For a given simulation the goal is to find the language \(\ell^*\) which
minimizes our cross-entropy objective and then check to see if that
language contains ambiguity.\par

Recall that language efficiency is both a function of the particular
semantic mappings induced by that language, the speaker and listener
agents (\(S\) and \(L\)), as well as the utterance (\(P(U)\)), meaning
(\(P(M)\)), and context priors (\(P(C)\)). Rather than assume particular
structure for our utterance, meaning, and context prior distributions,
for each simulation we generate \(P(U) \sim \text{Dir}(1, |U|)\),
\(P(M|C=c) \sim \text{Dir}(1, |M|)\) (a separete conditional
distribution over meanings for each context \(c\)), and
\(P(C) \sim \text{Dir}(1, |C|)\) where \(\text{Dir}(1, k)\) specifies
the uniform Dirichlet distribution over a \(k\)-dimensional probability
vector.\par

\subsubsection{Context}\label{context}

Following the argument given by Piantadosi et al. (2011), we want to
assess the impact of \emph{context} on our objective. To do so we
consider four conditions with \(n=500\) simulations each (that is, 500
unique sets of \(\{P(U), P(M|C), P(C)\}\). Our first is a
\emph{single-context} condition (\(|C|=1\)) -- there is a only a single
context describing \(P(M)\). Our second condition contains two-contexts
(\(|C| = 2\)) -- we consider efficiency under both \(P(M|C=c_1)\) as
well as \(P(M|C=c_2)\). The third and fourth condition correspond
accordingly with \(|C|=3\) and \(|C| = 4\), respectively.\par

\subsubsection{Baselines}\label{baselines}

For comparison, we also examine properties of optimal languages under
two additional objectives. Zipf (1949) proposed that the optimal speaker
language \(\ell_{speaker}^*\) would only optimize speaker effort and the
optimal listener language \(\ell_{listener}^*\) would only optimize
listener effort. We define these objectives using the first and second
half of equation 1 (see SI section 2.2. for more detail).\par

\subsection{Results and Discussion}\label{results-and-discussion}

Figure 1. plots the proportion of optimal languages under each
objective, as a function of context condition. The red line shows that
as the number of contexts increases, so does the probability that the
optimal language \(\ell^*_{cross}\) under our speaker-listener
cross-entropy objective contains ambiguity. We also plot the proportion
of optimal speaker-only \(\ell^*_{speaker}\) and listener-only
\(\ell^*_{listener}\) languages that contain ambiguity. In line with
Zipf's predictions, if languages are designed only to minimize speaker
effort then optimal languages will assign lower-cost forms multiple
meanings (all of the speaker-optimal languages contained ambiguity).
Likewise, if languages are designed only to minimize listener effort
then ambiguity should always be avoided (none of the listener-optimal
languages contained ambiguity).\par

We explored the degree to which ambiguity improved efficiency during
communication between rational, pragmatic agents. We showed that as
contextual information increased, the optimal languages under our
Cross-Entropy objective were more likely to have ambiguous lexical
items. Importantly, we did not see this effect under alternative
objectives which only emphasized speaker- or listener effort alone.\par

While our results indicate that ambiguity is an efficient property of
contextualized language these simulations assumed that speaker-listener
agents could always disambiguate items in context. That is, in our four
conditions both agents had perfect knowledge of the relevant conditional
distributions (\(P(M|C)\)). This assumption may be too strong for
describing much of day-to-day communication -- we seldom begin a
conversation with perfect knowledge of the current context of a
conversation. We explore a case in which the listener has imperfect
knowledge of context at the start of the conversation in Simulation
2.\par

\section{Simulation 2: Rational, pragmatic speakers use ambiguity
efficiently}\label{simulation-2-rational-pragmatic-speakers-use-ambiguity-efficiently}

We shift time-scales in Simulation 2 -- instead of examining the
objective landscape of a space of languages, we now consider a single
language that contains ambiguity and examine how that property is used
over a discourse. If context is not informative with respect to meaning
early in a discourse (position \(i\)), but is at later positions at
position (position \(i+k\)), then the speaker should avoid using
ambigous material early \(i\), only using it \(k\) steps later in the
conversation. In this toy example we can consider ``context'' as
analogous to a ``topic'' of conversation. Intuitively, we can imagine a
scenario in which a reader is beginning a newspaper article. While they
may have some knowledge about the article's topic (perhaps from the
title), they may not have complete knowledge of its contents, including
the persons or events involved. In this setting, using an ambiguous
pronoun early (position \(i\)) in the article can lead to
misunderstanding without sufficient context to support disambiguation.
But if by position \(i+k\) there is sufficient contextual information it
may be efficient for the writer to use ambiguous material.\par

\subsection{Simulation set-up}\label{simulation-set-up-1}

We consider consider a single language \(\ell\), which contains both
ambiguous and unambiguous utterances. The ambiguous utterances are less
costly than the unambiguous utterances. Crucially, we do not assume that
the listener knows the particular topic (\(c_{current}\)) of the
conversation \textit{a priori}. Rather, we assume that the listener has
knowledge of the set of possible topics \(C = \{c_1, \dots, c_k\}\), but
\textit{does not know which one is currently being used by the speaker}.
Formally, this means the listener does not have access to the correct
conditional distribution over meanings \(P(M|C=c_{current})\) at the
start of the discourse. In this setting, a speaker minimizing production
effort would like to use the low-cost material as frequently as
possible. However, this will incur a disambiguation cost to the listener
if they are uncertain of the current context \(c_{current}\) (topic of
conversation).\par 

Over the course of a discourse, \(D\) the listener tries to infer both
the current topic, \(c_{current}\), as well as the particular meaning
\(m\) of a given utterance \(u\). That is, we consider updated listener
and speakers (\(L(m, c|u,D;\ell)\), \(S(u|m,c,D;\ell)\)). Note that both
the speaker and listener can track the history of previous utterances
\(D\). Importantly, an agent can attempt to infer the current topic of
conversation \(c_{current}\) using the discourse history \(D\) (see SI
section 3 for more detail).\par

\begin{CodeChunk}
\begin{figure*}[h]

{\centering \includegraphics{figs/plot-optimal-use-1} 

}

\caption[(A) shows the empirical probability that our speaker used an ambiguous utterance as a function of discourse position]{(A) shows the empirical probability that our speaker used an ambiguous utterance as a function of discourse position. (B) shows speaker effort across the three models. (C) shows the Cross-Entropy objective under our three speaker models.}\label{fig:plot-optimal-use}
\end{figure*}
\end{CodeChunk}

We conduct \(N=600\) simulations, generating discourses of length
\(|D|=30\) utterances with three different speaker models (\(n=200\)
each). We consider a single language
\(\ell\)\footnote{See SI for the matrix notation of this langauge.} with
\(|U|=6\) and \(|M|=4\) in which two of the utterances \(u_5\) and
\(u_6\) are ambiguous and that
\(p(u_5) = p(u_6) > p(u_{1})=\dots=p(u_{4})\). That is, the two
ambiguous utterances (\(u_5\) and \(u_6\)) are less costly than the
non-ambiguous utterances. (Note that use of this particular language is
not essential -- the results are broadly generalizable languages that
contain ambiguity.)\par

\subsection{Speaker agents}\label{speaker-agents}

We consider three types of speaker models. In our
\textit{Full pragmatics} model, we consider a speaker agent who reasons
about her listener and also has complete recall of the set of utterances
in the discourse \(D\). This speaker believes that the listener may not
know the current topic \(c_{current}\) at the start of the discrouse,
but can infer it with enough data. In this way, contextual information
shared by the speaker and listener may increase over the discourse (as
the listeners estimate of the current topic (\(P(C=c|D)\)) comes in line
with the actual topic known to the speaker). We compare two baselines
models. The first, a \textit{Partial pragmatics} baseline describes a
speaker who reasons about a listener, but assumes they have no access to
discourse history. The second is a \textit{No pragmatics} speaker who
does not consider a listener at all, but produces utterances according
to the underlying langauge semantics (\(\ell\)) and topic probabilities
(\(p(M|C=c_{current}\)) (see SI section 3 for details).\par

\subsection{Hypotheses}\label{hypotheses}

We assess whether it is efficient to use low-cost, ambiguous material
when context is disambiguating. In the current setting, this corresponds
to speaker behavior in which ambiguous forms are avoided early in the
discourse, but prefered later in the discourse once the listener is
confident of the topic. We should expect this to be reflected in our
\textit{Full pragmatics} model. By contrast, our
\textit{Partial pragmatics} speaker does not believe that there is
sufficient contextual information for the listener to disambiguate her
utterances. This should lead to total avoidance of ambiguous material
over the discourse. Finally, our \textit{No pragmatics} speaker should
use low-cost material greedily since she does not consider a listener's
need to disambiguate whatsoever.\par

\subsection{Results and Discussion}\label{results-and-discussion-1}

Figure 2, (A) shows the empirical probability that a speaker uses an
ambiguous utterance as a function of discourse position. The
\text{No pragmatic} baseline uses ambiguous utterances frequently and at
a constant rate over the discourse and the \text{Partial pragmatic}
baseline avoids ambiguous utterances entirely (reasoning that the
utterances cannot be disambiguated). But, the \text{Full pragmatic}
model avoids ambiguous material at first, but employing them gradually
increasingly as the discourse proceeds. (B) shows the speaker-listener
Cross-Entropy objective. Note that the objective decreases for all three
models. This decrease in all three models is primarily driven by the
listener updating his belief about the actual topic
(\(P(C=c_{current}|D)\)). Of particular importance is the value of the
objective at the end of the discourse (position \(i=30\)). The
difference in the Cross-Entropy between the \text{Full Pragmatic} and
\text{Partial Pragmatic} models at the end of the discourse is driven by
the reduction in speaker costs described in (C). That is, while speaker
effort remains constant in both \textit{No pragmatic} and
\textit{Partial pragmatic} baselines, effort declines for the
\text{Full Pragmatic} model as they are increasingly able to use
ambiguous material later in the discourse.\par

In Simulation 2 we explored how ambiguity could be employed efficiently
during discourse. We assumed that the listener did not know the current
context (topic) \textit{a priori}, but could infer it from the discourse
history. We examined speaker production behavior, tracking use of
low-cost, but ambiguous material over the discourse. Early in the
discourse the \textit{Full pragmatic} speaker avoided using ambiguous
material, only using the material later when the context was known to
the listener -- it is useful to use low-cost material, but only when
context is disambiguating.

\section{General Discussion}\label{general-discussion}

How do the competing pressures of speakers and listeners give rise to
the distributional forms found in natural language? Zipf (1949) proposed
that the asymmetry between speaker and listener costs gives rise to a
range of properties at the level of the lexicon. We explored the
interactions of rational pragmatic agents as a framework for
understanding efficient language structure and use. We focused on an
argument on the communicative function of ambiguity (Piantadosi et al.,
2011). We derive a novel speaker-listener cross-entropy objective for
measuring the efficiency of a language in a reference game setting
showing that optimal languages are more likely to contain ambiguous
material when context is informative. Further we show that rational
pragmatic agents will \emph{use} ambiguous material efficiently, only
when such use is supported by context.\par

The framework employed here -- the analysis of the dynamics of repeated
reference games between rational pragmatic agents -- provides a tool for
the analysis of functionalist theories, allowing linkage across
timescales of analysis from conversation (Genzel \& Charniak, 2002; \&.
J. Levy R., 2007) to typology (Regier et al., 2015). In theory, this
framework provides extensive opportunity to vary properties of the
languages and agents, however we have focused on only a small subset
here. For example, examining how increasing or decreasing
\textit{rationality} (instantiated through the RSA \(\alpha\) parameter)
or \textit{pragmatics} (instantiated through the RSA recursion depth)
impacts these results is a clear avenue for future work. An important
limitation of the current work is the broad assumption that language can
be described through such reference games. The degree to which
properties of non-referential language can be captured in such a
framework remains an important target for future work.

\vspace{1em}
\fbox{\parbox[b][][c]{7.3cm}{\centering All simulation and analysis code are available in the public repository for the project: (link will be available upon acceptance)}}
\vspace{1em} \noindent

\section{References}\label{references}

\setlength{\parindent}{-0.1in} \setlength{\leftskip}{0.125in} \noindent

\hypertarget{refs}{}
\hypertarget{ref-BennettGoodman2015a}{}
Bennett, E., \& Goodman, N. (2018). Extremely costly intensifiers are
stronger than quite costly ones. \emph{Cognition}.

\hypertarget{ref-Chomsky2002a}{}
Chomsky, N. (2002). An interview on minimilism. In \emph{N chomsky, on
nature and language}.

\hypertarget{ref-FerreriCancho2018a}{}
Ferrer-i-Cancho, R. (2018). Optimization models of natural
communication. \emph{Journal of Quantitative Linguistics}, \emph{25
(3)}, 207--237.

\hypertarget{ref-GenzelCharniak2002a}{}
Genzel, D., \& Charniak, E. (2002). Entropy rate constancy in text. In
\emph{Proceedings of the 40th annual meeting on association for
computational linguistics}.

\hypertarget{ref-GoodmanFrank2016a}{}
Goodman, N., \& Frank, M. (2016). Pragmatic language interpreation as
probabilistic inference. \emph{Trends in Cognitive Sciences},
\emph{20(11)}, 818--829.

\hypertarget{ref-Grice1975a}{}
Grice, P. H. (1975). Logic and conversation.

\hypertarget{ref-Hale2001a}{}
Hale, J. (2001). A probabilistic earley parser as a psycholinguistic
model. In \emph{Proceedings of the naacl}. 159-166.

\hypertarget{ref-HawkinsFrankeSmithGoodman2018a}{}
Hawkins, R., Franke, M., Smith, K., \& Goodman, N. (2018). Emerging
abstractions: Lexical conventions are shaped by communicative context.
In \emph{Proceedings of the 40th annual conference of the cognitive
science society}.

\hypertarget{ref-KirbyGriffithsSmith2014a}{}
Kirby, S., Griffiths, T., \& Smith, K. (2014). Iterated learnign and the
evoluation of language. \emph{Current Opinion in Neurobiology},
\emph{28}, 108--114.

\hypertarget{ref-LevyJaeger2007a}{}
Levy, \&. J., R. (2007). Speakers optimize information density through
syntactic reduction. In \emph{Proceedings of the twentieth annual
conference on neural information processing systems}.

\hypertarget{ref-Levy2008a}{}
Levy, R. (2008). Expectation-based syntactic comprehension.
\emph{Cognition}, \emph{106(3)}, 1126--1177.

\hypertarget{ref-Piantadosi2011a}{}
Piantadosi, S., Tily, H., \& Gibson, E. (2011). The communicative
function of ambiguity in language. \emph{Cognition}, \emph{122},
280--291.

\hypertarget{ref-RegierKempKay2015a}{}
Regier, T., Kemp, C., \& Kay, P. (2015). Word meanings across languages
support efficient communication. In B. M. \& W. O'Grady (Ed.), \emph{The
ahndbook of language emergence (pp. 237-263)}. Hoboken, NJ:
Wiley-Blackwell.

\hypertarget{ref-WasowPerforsBeaver2005a}{}
Wasow, T., Perfors, A., \& Beaver, D. (2005). The puzzle of ambiguity.
In CSLI (Ed.), \emph{Morphology and the web of grammar: Essays in memory
of steven g lapointe}. Stanford, CA: McGraw-Hill.

\hypertarget{ref-Zipf1949a}{}
Zipf, G. (1949). \emph{Human behavior and the principle of least
effort}. New York, NY: Prentice-Hall.

\bibliographystyle{apacite}


\end{document}
