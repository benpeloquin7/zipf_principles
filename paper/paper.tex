% Template for Cogsci submission with R Markdown

% Stuff changed from original Markdown PLOS Template
\documentclass[10pt, letterpaper]{article}

\usepackage{cogsci}
\usepackage{pslatex}
\usepackage{float}
\usepackage{caption}

% amsmath package, useful for mathematical formulas
\usepackage{amsmath}

% amssymb package, useful for mathematical symbols
\usepackage{amssymb}

% hyperref package, useful for hyperlinks
\usepackage{hyperref}

% graphicx package, useful for including eps and pdf graphics
% include graphics with the command \includegraphics
\usepackage{graphicx}

% Sweave(-like)
\usepackage{fancyvrb}
\DefineVerbatimEnvironment{Sinput}{Verbatim}{fontshape=sl}
\DefineVerbatimEnvironment{Soutput}{Verbatim}{}
\DefineVerbatimEnvironment{Scode}{Verbatim}{fontshape=sl}
\newenvironment{Schunk}{}{}
\DefineVerbatimEnvironment{Code}{Verbatim}{}
\DefineVerbatimEnvironment{CodeInput}{Verbatim}{fontshape=sl}
\DefineVerbatimEnvironment{CodeOutput}{Verbatim}{}
\newenvironment{CodeChunk}{}{}

% cite package, to clean up citations in the main text. Do not remove.
\usepackage{apacite}

% KM added 1/4/18 to allow control of blind submission


\usepackage{color}

% Use doublespacing - comment out for single spacing
%\usepackage{setspace}
%\doublespacing


% % Text layout
% \topmargin 0.0cm
% \oddsidemargin 0.5cm
% \evensidemargin 0.5cm
% \textwidth 16cm
% \textheight 21cm

\title{The interactions of rational, pragmatic agents\\
lead to efficient language structure and use}


\author{{\large \bf Benjamin N. Peloquin} \\ \texttt{bpeloqui@stanford.edu} \\ Department of Psychology \\ Stanford University \And {\large \bf Noah D. Goodman} \\ \texttt{ngoodman@stanford.edu} \\ Department of Computer Science \\ Stanford University \And {\large \bf Michael C. Frank} \\ \texttt{mcfrank@university.edu} \\ Department of Psychology \\ Stanford University}

\begin{document}

\maketitle

\begin{abstract}
Despite their diversity, languages around the world share a consistent
set of properties and distributional regularities. For example, the
distribution of word frequencies, the distribution of syntactic
dependency lengths, and the presence of ambiguity are all remarkably
consistent across languages. We discuss a framework for studying how
these system-level properties emerge from local, in-the-moment
interactions of rational, pragmatic speakers and listeners. To do so, we
derive a novel objective function for measuring the communicative
efficiency of linguistic systems in terms of the interactions of
speakers and listeners. We examine the behavior of this objective in a
series of simulations focusing on the communicative function of
ambiguity in language. These simulations suggest that rational pragmatic
agents will produce communicatively efficient systems and that
interactions between such agents provide a framework for examining
efficient properties of language structure and use more broadly.

\textbf{Keywords:}
Communicative efficiency, Rational Speech Act theory, computational
modeling, information theory, agent-based simulation
\end{abstract}

\section{Introduction}\label{introduction}

Why do languages look the way they do? Zipf (1949) proposed that
distributional properties found in natural language were evidence of
speaker-listener effort minimization. In his own words, ``we are arguing
that people do in fact act with a maximum economy of effort, and that
therefore in the process of speaking-listening they will automatically
minimize the expenditure of effort.'' Evidence for this claim has been
largely derived at the level of the lexicon. Zipf argued that the
particular relationship between a word's frequency and its rank, length,
and denotation size could be explained as an emergent property of
speaker-listener effort minimization. \par

Zipf articulated what is now considered a \emph{functionalist} approach
to language science -- analyzing language structure and use in terms of
efficiency. Such an approach might reframe our opening question as
follows: how does having property \textit{x} make using language
\(\ell\) more or less useful for communication? Among others, this
efficiency-based framing has produced a rich set of theoretical and
empirical targets exploring semantic typology (Regier, Kemp, \& Kay,
2015), properties such as ambiguity (Piantadosi, Tily, \& Gibson, 2011)
and compositionality (Kirby, Griffiths, \& Smith, 2014), and the
efficient use of reduction and redundancy in production (Genzel \&
Charniak, 2002; Levy \& Jaeger, 2007). These projects typically employ
information-theoretic principles to characterize efficiency in natural
or simulated language corpora.\par

We adopt a different approach in this work in two ways. First we derive
a measure of efficiency from first principles of
\textit{efficient language use} and show how this can lead to
communicatively efficient systems. Second we demonstrate how assumptions
about language-users impacts whether language properties are used
efficiently. In this way, we bridge questions of design and use in a
single framework.\par

Functionalist theories commonly frame language efficiency in terms of a
fundamental effort-assymetry underlying everyday communication: what is
``hard'' for a speaker is likely different than what is ``hard'' for a
listener. Zipf described this as follows -- purely from the standpoint
of speaker effort, an optimal language \(\ell_{speaker}^*\) would tend
toward a vocabulary of a single, low-cost word. Given such a language,
the full set of potential meanings would be conveyed using only that
word, i.e. \(\ell_{speaker}^*\) would be fully ambiguous and all
possible meanings would need to be disambiguated by a listener. From the
standpoint of listener effort, an optimal language \(\ell_{listener}^*\)
would map all possible meanings to distinct words, removing a listener's
need to disambiguate. In this example, speaker effort is related to
\emph{production cost} and listener effort to \emph{understanding or
disambiguation cost}. Clearly, natural languages fall between the two
extremes of \(\ell_{speaker}^*\) and \(\ell_{listener}^*\). Zipf
proposed that the particular lexicon-level properties he observed were a
result of these competing forces -- the pressure to jointly minimize
speaker and listener effort.\par

But how does this optimization take place? The example given by Zipf
(1949) describes local, communicative interactions in terms of a
\textit{reference game}. Speakers intend to refer to some object in the
world \(m\). They choose some utterance \(u\) to transmit this intended
meaning, \(u \rightarrow m\). The listener attempts to reconstruct this
intended meaning given the transmitted utterance, \(m \rightarrow u\).
Other projects have assumed this basic reference game setting
(Piantadosi et al., 2011; Regier et al., 2015) and this simplification
of the communicative act has proven productive in theoretical
(Ferrer-i-Cancho, 2018), simulation-based (Kirby et al., 2014) and
empirical explorations (Hawkins, Franke, Smith, \& Goodman, 2018) of
efficient language structure and use.\par

Adopting reference games as a basic unit of analysis suggests that
optimization may take place at the level of conversation. Importantly,
Zipf's conception of speaker and listener effort may be connected to how
language is used; in particular, whether interlocutors engage in
pragmatic reasoning during conversation. Under a Gricean treatment of
pragmatics, speakers and listeners follow a set of conversational maxims
in which they cooperate to transfer information (Grice, 1975). These
maxims appear to emerge from efficiency concerns, however (Horn, 1984).
We formalize this connection -- showing how system-level efficiencies
can emerge from local interaction behavior of pragmatic agents. Our
claim is that to understand an ``efficient'' property of a system it is
essential that we consider how that property is \emph{used}
efficiently.\par

We provide a case study for this approach, in which functionalist
regularities emerge from the dynamics of pragmatic communication. We
choose a property of languages that could, in principle, vary freely,
but shows strong regularities across languages. The explanadum is why
this regularity holds. We examine ambiguity as our property, extending
ideas by Piantadosi et al. (2011). We define a novel measure of
efficiency that depends on the interactional behavior of speaker and
listener agents. We adopt the reference game as our primary unit of
interaction and model language users with the Rational Speech Act (RSA)
framework -- a computional model of language use, which is nicely
attested by experimental data on interaction. Using these basic
ingredients, we show that the property of interest (ambiguity) is
prevalent in languages that optimize our measure of efficiency. Further,
we show how ambiguity is \textit{used} efficiently during local,
in-the-moment interactions. The contributions of this work are twofold
-- we derive a novel objective for measuring linguistic efficiency and
also show how the reference game framework, in combination with formal
models of communication, can be used to connect ideas about system-level
efficiencies to in-the-moment language use.\par

\section{Exploring efficient language design and use in rational
pragmatic
agents}\label{exploring-efficient-language-design-and-use-in-rational-pragmatic-agents}

\begin{CodeChunk}
\begin{figure}[H]

{\centering \includegraphics{figs/plot-reference-game-1} 

}

\caption[An example reference game with associated literal semantics (in our terminology a ``language'')]{An example reference game with associated literal semantics (in our terminology a ``language'').}\label{fig:plot-reference-game}
\end{figure}
\end{CodeChunk}

\subsubsection{Reference games}\label{reference-games}

Zipf's example of optimal speaker- and listener-languages took the form
of a reference game. We adopt that formulation here, assuming these
communication games as our basic unit of analysis. In this framework,
speakers and listeners are aware of a set of objects \(M\)
(\emph{meanings}) and are knowledgeable about the set of possible
signals \(U\) (\emph{utterances}) that can be used to refer to a given
meaning (see Figure 1). Utterances may have different relative costs,
operationalized via a prior over utterances \(P(U)\). Similarly,
meanings differ in the relative degree to which they need to be talked
about, operationalized as a prior over meanings
\(P(M)\)\footnote{The prior over meanings are analogous to the \textit{need probabilities} assumed in previous work (Regier, Kemp \& Kay (2015).}.
We consider a set of contexts \(C\) with an associated prior \(P(C)\).
Each context \(c\in C\) describes a different distribution over meanings
e.g. \(p(M|C=c_i) \neq p(M|C=c_j)\). Finally, we consider a set of
communicative events \(e \in E\) where \(<u, m, c> = e\) is an
utterance-meaning-context triple.\par

\subsubsection{Languages}\label{languages}

A language \(\ell\) defines the set of semantic mappings between
utterance and meanings. For example, Figure 1 contains four utterances
\(U = \{\text{blue}, \text{green}, \text{square}, \text{circle}\} \text{ and three meanings }M = \{\text{green-square}, \text{blue-square}, \text{green-circle}\}\).
The boolean matrix describes the literal semantics of the language. We
define a language as ``ambiguous'' if there is some utterance
\(u \in U\) which can apply to multiple meanings (i.e.
\(|[[u_i]]| > 1\))\footnote{We use double brackets $[[\dots]]$ to represent denotation.}.
In Figure 1 both the words ``square'' and ``green'' are ambigous so we
would say that \(\ell\) contains ambiguity.

\subsubsection{Speakers and listeners}\label{speakers-and-listeners}

The Rational Speech Act framework (RSA) is a computational-level theory
of pragmatic language use, which has produced good fit to human
communication behavior across a range of language phenomena (Frank \&
Goodman, 2012; Goodman \& Frank, 2016). RSA is a formalization of
essential Gricean pragmatic principles -- agents reason about one
another and their shared context (Grice, 1975). We adopt RSA as our
representational framework to model Gricean (rational and pragmatic)
speaker-listeners in the reference game setting (see
\href{https://github.com/benpeloquin7/zipf_principles/blob/master/paper/supplementary_materials.pdf}{SI}
and Goodman \& Frank (2016) for a review).\par

An RSA \emph{speaker agent} defines a conditional distribution over
utterances, mapping from intended meanings \(M\) to utterances \(U\)
using \(\ell\) in a given context \(c\). That is, a speaker defines
\(P_{speaker}(u|m, c; \ell)\). We will use \(S(u|m, c; \ell)\) as
short-hand throughout. A \emph{listener agent} defines a conditional
distribution over meanings, mapping from utterances \(U\) to meanings
\(M\) using \(\ell\) in a given context \(c\) (i.e. \(L(m|u, c;\ell)\)).
Note that both speakers and listeners can induce joint distributions
over utterace-meaning pairs, although, importantly, these distributions
may differ: \[P_{speaker}(u, m | c; \ell) = S(u|m, c; \ell)p(m|c)\]
\[P_{listener}(u, m| c; \ell) = L(m|u, c; \ell)p(u|c)\]

\section{Zipfian objective for linguistic system
efficiency}\label{zipfian-objective-for-linguistic-system-efficiency}

Zipf (1949) proposed that the particular distributional properties found
in natural language emerge from competing speaker and listener
pressures. We operationalize this objective in equation (1) -- the
efficiency of a linguistic system \(\ell\) being used by speaker and
listener agents \(S\) and \(L\) is the sum of the expected speaker and
listener effort to communicate over all possible communicative events
\(e \in E\).\par

\begin{equation}
\begin{split}
  \text{Efficiency}(S, L, \ell) = \mathbb{E}_{e \sim P(E)}[\text{speaker effort}] \\+ \mathbb{E}_{e \sim P(E)}[\text{listener effort}]
\end{split}
\end{equation}

We assume that speaker effort is related to the surprisal of an
utterance in a particular
context\footnote{In the current set of simulations we consider utterances costs as independent from context (ie. $p(u|c)p(c)=p(u)p(c)$).}
-- intuitively, the number of bits needed to encode the utterance \(u\).
This particular formalization of speaker-cost is general enough to
accommodate a range of cost instantiations, such as production
difficulty via articulation effort, cognitive effort related to lexical
access, or others (Bennett \& Goodman, 2018).\par

\[\text{speaker effort} = -log_2(p(u|c))\]

We assume listener effort is the semantic surprisal of a meaning given
an utterance. This operationalization of listener effort is intuitively
related to existing work in sentence processing in which word
comprehension difficulty is proportional to surprisal (Hale, 2001; Levy,
2008).

\[\text{listener effort} = -log_2(L(m|u, c; \ell))\]

Importantly, we assume that events \(e = <u, m, c>\) are sampled
according the to following generative model -- some context occurs in
the world with probability \(P(C=c)\). Within this context, an object
\(m\) occurs with probability \(p(m|c)\). The speaker attempts to refer
to that object by sampling from her conditional distribution
\(S(u|m, c; \ell)\) (i.e. \(e \sim p(c)p(m|c)S(u|m, c; \ell)\)). From
the ingredients described above, it is possible to derive the following
objective between the speaker and listener distributions (see
\href{https://github.com/benpeloquin7/zipf_principles/blob/master/paper/supplementary_materials.pdf}{SI 2.1}
for a complete derivation).

\begin{equation}
\begin{split}
  = \mathbb{E}_{c \sim P(C)}[H_{cross}(P_{speaker}, P_{listener} | c; \ell)]\\
\end{split}
\end{equation}

From an information-theoretic perspective this objective is intuitive:
\(H_{cross}\) denotes the Cross-Entropy (CE), a measure of dissimilarity
between two distributions -- the average number of bits required to
communicate under one distribution, given that the ``true'' distribution
differs. In our case, we have an expectation over this term -- the
expected difference between the distributions assumed by the speaker
\(P_{speaker}\) and listener \(P_{listener}\) given a set of contexts
\(C\)\footnote{Note that in the single context case $|C|=1$ this objective is simply the speaker-listener Cross-Entropy.}.
In other words, an ``efficient'' language \(\ell\) minimizes the
distance between what speakers and listeners think.

\section{Simulating the communicative function of
ambiguity}\label{simulating-the-communicative-function-of-ambiguity}

The task of understanding language is marked by a frequent need to
handle various forms of ambiguity: lexical, syntactic, among others
(Wasow, Perfors, \& Beaver, 2005). The ubiquity of this property,
however, has been argued to provide evidence that languages have not
been optimized for communication (Chomsky, 2002).

Piantadosi et al. (2011) argue just the opposite, claiming that
ambiguity is an \emph{efficient} property of any communication system in
which \emph{communication is contextualized}. Simply put, it is useful
to have a language that re-uses low-cost material (has ambiguity) so
long as the cost of disambiguating the material is low. In particular,
context (or common ground) can provide useful information for
disambiguation.\par

As an example, say we have two objects (\(m_1\) and \(m_2\)), two
utterances (\(u_1\) and \(u_2\)), which differ in cost, and two
languages (\(\ell_1\) and \(\ell_2\)), which describe different
utterance-meaning mappings. In language \(\ell_1\), the low-cost \(u_1\)
can be used to refer to both \(m_1\) and \(m_2\)
(\([[u_1]]_{\ell_1} = \{m_1, m_2 \}\)), but the high-cost \(u_2\) cannot
be used at all (\([[u_2]]_{\ell_1} = \emptyset\)). By contrast, in
language \(\ell_2\), \(u_1\) can only refer to \(m_1\) and \(u_2\) can
only refer to \(m_2\) (\([[u_1]]_{\ell_2} = \{m_1\}\) and
\([[u_2]]_{\ell_2} = \{m_2 \}\)). While it is cheaper for a speaker to
use \(\ell_1\) (because speaking is always low-cost), it is more
difficult for a listener (because \(u_1\) is ambiguous). Crucially,
however, if context is disambiguating then the speaker can use \(u_1\)
to refer to either \(m_1\) or \(m_2\) and \(\ell_1\) should be preferred
to \(\ell_2\).\par

In the following experiments we explore two aspects of Piantadosi's et
al.'s claim. In Simulation 1, we examine the efficient language
\emph{structure} aspect of their claim, exploring when the optimal
linguistic system \(\ell^*\) is most likely to contain ambiguous
expressions. In Simulation 2, we explore an efficient language
\emph{use} aspect of the claim -- at what point in a conversation is it
useful for a speaker to use ambiguous expressions?\par

\begin{CodeChunk}
\begin{figure*}[h]

{\centering \includegraphics{figs/plot-optimal-langs-1} 

}

\caption[Panel (A) Vertical axis shows the proportion of optimal languages containing ambiguity]{Panel (A) Vertical axis shows the proportion of optimal languages containing ambiguity. Horizontal axis shows the number of context-sizes (1-4). Optimal language under CE objective (red). Speaker-optimal (blue). Listener-optimal (green). Error bars represent 95 percent confidence intervals. Panel (B), example CE-optimal language (ambiguous) under in a four-condition context. Panel (C), example CE-optimal language (unambiguous) in a single-condition context.}\label{fig:plot-optimal-langs}
\end{figure*}
\end{CodeChunk}

\section{Simulation 1: Optimal languages contain ambiguity when context
is
informative}\label{simulation-1-optimal-languages-contain-ambiguity-when-context-is-informative}

We show that ambiguity is an efficient property under our CE objective
in the reference game setting. We proceed by generating languages with
different amounts of contextual support (varying the size of \(|C|\)).
We search the space of languages, examining whether ones which minimize
our objective contain ambiguity. If context leads to more efficient
communication, then optimal languages should be more likely to be
ambiguous as the amount of context increases.\par

\subsection{Simulation set-up}\label{simulation-set-up}

We conduct \(N=2000\) simulations. For each simulation we enumerate the
set of \emph{valid} languages in which \(|U|=|M|=4\) (\(U\) is our set
of utterances and \(M\) our set of meanings). Recall that languages are
boolean matrices and a language \(\ell \in L\) is ``valid'' so long as
each possible meaning in \(m \in M\) can be referred to by at least one
form \(u \in U\) (every column of \(\ell\) has some non-zero assignment)
and each form maps to at least one meaning (every row has some non-zero
assignment). For a given simulation, the goal is to find the language
\(\ell^*\) which minimizes our objective and then check to see if that
language contains ambiguity.\par

We define language efficiency as a function of the particular semantic
mappings induced by that language, the speaker and listener agents
(\(S\) and \(L\)), as well as the utterance (\(P(U)\)), meaning
(\(P(M)\)), and context priors (\(P(C)\)). Rather than assume particular
structure, for each simulation we generate
\(P(U) \sim \text{Dir}(1, |U|)\), \(P(M|C=c) \sim \text{Dir}(1, |M|)\)
(a separate conditional distribution over meanings for each context
\(c\)), and \(P(C) \sim \text{Dir}(1, |C|)\), where \(\text{Dir}(1, k)\)
specifies the uniform Dirichlet distribution over a \(k\)-dimensional
probability vector.\par

\begin{CodeChunk}
\begin{figure*}[h]

{\centering \includegraphics{figs/plot-optimal-use-1} 

}

\caption[(A) shows the empirical probability that our speaker used an ambiguous utterance as a function of discourse position]{(A) shows the empirical probability that our speaker used an ambiguous utterance as a function of discourse position. (B) shows speaker effort across the three models. (C) shows the Cross-Entropy objective under our three speaker models. Error bars represent 95 percent confidence intervals.}\label{fig:plot-optimal-use}
\end{figure*}
\end{CodeChunk}

\subsubsection{Context}\label{context}

We want to assess the impact of \emph{context} on the presence of
ambiguity in optimal languages. To do so we consider four conditions
with \(n=500\) simulations each (that is, 500 unique sets of
\(\{P(U), P(M|C), P(C)\}\). Our first is a \textit{one-context}
condition (\(|C|=1\)) -- only a single distribution over meanings
\(P(M)\). In our \textit{two-context} condition (\(|C| = 2\)), we
consider efficiency under both \(P(M|C=c_1)\) as well as \(P(M|C=c_2)\).
\textit{Three-} and \textit{four-context} conditions corresponding
accordingly.\par

\subsubsection{Baselines}\label{baselines}

For comparison, we also examine properties of optimal languages under
two additional objectives. Zipf (1949) proposed that the speaker-optimal
language \(\ell_{speaker}^*\) would minimize speaker effort and the
listener-optimal language \(\ell_{listener}^*\) would minimize listener
effort. We define these objectives using the first and second half of
equation 1 (see
\href{https://github.com/benpeloquin7/zipf_principles/blob/master/paper/supplementary_materials.pdf}{SI 2.2.}).\par

\subsection{Results and Discussion}\label{results-and-discussion}

In Simulation 1 we explored the degree to which ambiguity is an
efficient property of languages when communication is contextualized.
Figure 2, panel (A) plots the proportion of optimal languages under each
objective as a function of number of contexts. The red line shows that
as the number of contexts increases, so does the probability that an
optimal language \(\ell^*_{cross}\) contains ambiguity (at least one
utterance maps to two meanings) under our CE objective. For comparison
we also plot the proportion of speaker-optimal \(\ell^*_{speaker}\)
(blue line) and listener-optimal \(\ell^*_{listener}\) (green line)
languages that contain ambiguity. In line with Zipf's predictions, if
languages are designed only to minimize speaker effort then optimal
languages always contain ambiguity. If languages are designed to
minimize listener effort then ambiguity is always avoided.\par

While our results indicate that ambiguity is an efficient property of
contextualized language, these simulations assumed that agents had
perfect knowledge of the relevant conditional distributions
(\(P(M|C)\)). This assumption may be too strong for describing much of
day-to-day communication -- we seldom interact with others with perfect
knowledge of the current context (or topic) at the start of a
conversation. To explore how ambiguity may be \textit{used} efficiently
in our framework, we next examine a case in which the listener has
imperfect knowledge of context at the start of the conversation, but may
infer it from the discourse history.\par

\section{Simulation 2: Rational, pragmatic speakers use ambiguity
efficiently}\label{simulation-2-rational-pragmatic-speakers-use-ambiguity-efficiently}

In Simulation 1 we showed that efficiency defined in terms of pragmatic
agents leads to a preference for languages that contain ambiguity. In
Simulation 2 we assume a single fixed language \(\ell\), which contains
ambiguity, and instead vary the types of agents using \(\ell\). We will
show that efficient \emph{use} of ambiguity depends on an agent's
ability to use context for disambiguation. This instantiates an on-line
perspective on the communicative function of ambiguity theory. More
generally, it demonstrates how the reference game framework can be used
to asses questions of efficient language use (current simulation) as
well as language design (Simulation 1).\par

Imagine a scenario in which a reader is beginning a news article. While
they may have some knowledge about the article's topic (perhaps from the
title), they may not have complete knowledge of its contents, including
the persons or events involved. In this setting, using a low-cost, but
ambiguous referring expression (say a pronoun like ``he'') early may
lead to misunderstanding if context is not informative. But, if by a
later position enough contextual information has accumulated, it may be
efficient to use the ambiguous expression. We pursue this general
framework in Simulation 2 -- examining when in a discourse using
ambigity is efficient. Note that we will consider ``context'' as
analogous to a ``topic'' of conversation. \par

\subsection{Simulation set-up}\label{simulation-set-up-1}

We consider a single language \(\ell\), which contains both ambiguous
and unambiguous utterances. We assume ambiguous utterances are lower
cost. Crucially, we do not assume that the listener knows the particular
topic (\(c_{current}\)) of the conversation \textit{a priori}. Rather,
that the listener has knowledge of the set of possible topics
\(C = \{c_1, \dots, c_k\}\), but does not know which one is currently
being used by the speaker. Formally, this means the listener does not
have access to the correct conditional distribution over meanings
\(P(M|C=c_{current})\) at the start of the discourse.\par 

Over the course of a discourse \(D\), the listener tries to infer both
the current topic, \(c_{current}\), as well as the particular meaning
\(m\) of a given utterance \(u\). That is, we consider agents who can
track the history of previous utterances \(D\). Importantly, an agent
can attempt to infer the current topic of conversation \(c_{current}\)
using the discourse history \(D\).\par

We conduct \(N=600\) simulations, generating discourses of length
\(|D|=30\) utterances, comparing three speaker models (\(n=200\) each).
We consider a single language
\(\ell\)\footnote{See SI for the matrix notation of this langauge.} with
\(|U|=6\) and \(|M|=4\) in which two of the utterances are ambiguous and
lower cost than the unambiguous utterances. (Note that use of this
particular language is not essential -- the results are broadly
generalizable to languages that contain ambiguity, but exploring this
space is computationally expensive.)\par

\subsection{Speaker agents}\label{speaker-agents}

We vary the degree to which agents can use context for disambiguation.
We consider three types of speaker models. Our \textit{Full pragmatics}
agent, models a speaker who reasons about her listener and also has
complete recall of the set of utterances in the discourse \(D\). This
speaker believes that the listener may not know the current topic
\(c_{current}\) at the start of the discourse, but can infer it over the
discourse. We compare two baseline models. The first, a
\textit{Partial pragmatics} baseline describes a speaker who reasons
about a listener, but assumes they have no access to the discourse
history. The second, a \textit{No pragmatics} baseline speaker does not
consider a listener at all, but produces utterances according to the
underlying langauge semantics (\(\ell\)) and topic probabilities
(\(p(M|C=c_{current}\)) (see
\href{https://github.com/benpeloquin7/zipf_principles/blob/master/paper/supplementary_materials.pdf}{SI 3}).\par

\subsection{Hypotheses}\label{hypotheses}

We are interested in how each speaker-model uses ambiguity over the
discourse. A speaker strategy that is mutually efficient for both agents
should avoid ambiguity until sufficient contextual information has
accumulated. We should expect this to be reflected in our
\textit{Full pragmatics} model who reasons about the listener and
discourse history. By contrast, a speaker-optimal model who does not
consider the listener should greedily use ambiguous utterances
(\textit{No Pragmatics} model), while a listener-optimal model should
avoid ambiguity entirely (\textit{Partial pragmatics} model).\par

\subsection{Results and Discussion}\label{results-and-discussion-1}

Figure 2 (A) shows the empirical probability that a speaker uses an
ambiguous utterance as a function of discourse position. The
\textit{No pragmatic} baseline uses ambiguous utterances frequently and
at a constant rate over the discourse. The \textit{Partial pragmatic}
baseline avoids ambiguous utterances entirely. But, the
\textit{Full pragmatic} model avoids ambiguous material only at the
start of the discourse, employing it increasingly as the discourse
proceeds. Panel (C) tracks our CE objective for each model over the
discourse. Note that the objective decreases for all three models,
primarily driven by the listener updating his belief about the actual
topic (\(P(C=c_{current}|D)\)). However, the objective declines more
quickly under the \textit{Full} and \textit{Partial} pragmatic speakers
as listener agents are better able to infer the correct context.
Additionally, the difference in CE between the \textit{Full}- and
\textit{Partial pragmatic} models at the end of the discourse is driven
by the reduction in speaker costs (see panel B). While speaker effort
remains constant in both \textit{No pragmatic} and
\textit{Partial pragmatic} baselines, effort declines in the
\textit{Full pragmatic} model as she increasingly relies on ambiguous
material later in the discourse.\par

\section{General Discussion}\label{general-discussion}

How do the competing pressures of speakers and listeners give rise to
the distributional regularities found in natural language? Zipf (1949)
proposed that the asymmetry between speaker and listener costs gives
rise to a range of properties at the level of the lexicon. We explored
the interactions of rational pragmatic agents as a framework for
understanding efficient language structure and use. We focused on an
argument on the communicative function of ambiguity (Piantadosi et al.,
2011), deriving a novel speaker-listener Cross-Entropy objective for
measuring the efficiency of linguistic systems from first principles of
efficient language use. In Simulation 1 we showed that optimal languages
are more likely to contain ambiguous material when context is
informative. In Simulation 2 we showed how rational pragmatic agents use
ambiguous material efficiently in conversation.\par

A limitation of the current work is an analysis of exactly how the CE
objective compares to existing measures. For example, previous work has
described competing speaker-listener pressures in terms of a trade-off
of simplicity and informativeness (Kemp \& Regier, 2012) or expressivity
and compressibility (Smith, Tamariz, \& Kirby, 2013) to explain
linguistic regularities. Future work should assess the degree to which
we can derive the same properties as previous studies using our current
framework. In line with this idea, a primary focus of the current work
is to provide a case-study on the productivity of assessing
functionalist theories in repeated reference games between rational
pragmatic agents. We hope this framework can serve as a domain general
tool to assess the range of functionalist theories examining efficient
language-structure and use.\par

\vspace{1em}

\fbox{\parbox[b][][c]{7.3cm}{\centering {Code and SI are available at this \href{https://github.com/benpeloquin7/zipf_principles/}{link}.}}}
\vspace{1em} \noindent

\section{References}\label{references}

\setlength{\parindent}{-0.1in} \setlength{\leftskip}{0.125in} \noindent

\hypertarget{refs}{}
\hypertarget{ref-BennettGoodman2015a}{}
Bennett, E., \& Goodman, N. (2018). Extremely costly intensifiers are
stronger than quite costly ones. \emph{Cognition}.

\hypertarget{ref-Chomsky2002a}{}
Chomsky, N. (2002). An interview on minimilism. In \emph{N chomsky, on
nature and language}.

\hypertarget{ref-FerreriCancho2018a}{}
Ferrer-i-Cancho, R. (2018). Optimization models of natural
communication. \emph{Journal of Quantitative Linguistics}, \emph{25
(3)}, 207--237.

\hypertarget{ref-FrankGoodman2012a}{}
Frank, M., \& Goodman, N. (2012). Predicting pragmatic reasoning in
language games. \emph{Science}, \emph{336}, 998.

\hypertarget{ref-GenzelCharniak2002a}{}
Genzel, D., \& Charniak, E. (2002). Entropy rate constancy in text. In
\emph{Proceedings of the 40th annual meeting on association for
computational linguistics}.

\hypertarget{ref-GoodmanFrank2016a}{}
Goodman, N., \& Frank, M. (2016). Pragmatic language interpretation as
probabilistic inference. \emph{Trends in Cognitive Sciences},
\emph{20(11)}, 818--829.

\hypertarget{ref-Grice1975a}{}
Grice, P. H. (1975). Logic and conversation.

\hypertarget{ref-Hale2001a}{}
Hale, J. (2001). A probabilistic earley parser as a psycholinguistic
model. In \emph{Proceedings of the naacl}. 159-166.

\hypertarget{ref-HawkinsFrankeSmithGoodman2018a}{}
Hawkins, R., Franke, M., Smith, K., \& Goodman, N. (2018). Emerging
abstractions: Lexical conventions are shaped by communicative context.
In \emph{Proceedings of the 40th annual conference of the cognitive
science society}.

\hypertarget{ref-KempRegier2012a}{}
Kemp, C., \& Regier, T. (2012). Kinship categories across languages
reflect general communicative principles. \emph{Science}, \emph{336},
1049--1054.

\hypertarget{ref-KirbyGriffithsSmith2014a}{}
Kirby, S., Griffiths, T., \& Smith, K. (2014). Iterated learnign and the
evoluation of language. \emph{Current Opinion in Neurobiology},
\emph{28}, 108--114.

\hypertarget{ref-Levy2008a}{}
Levy, R. (2008). Expectation-based syntactic comprehension.
\emph{Cognition}, \emph{106(3)}, 1126--1177.

\hypertarget{ref-LevyJaeger2007a}{}
Levy, R., \& Jaeger, T. (2007). Speakers optimize information density
through syntactic reduction. In \emph{Proceedings of the twentieth
annual conference on neural information processing systems}.

\hypertarget{ref-Piantadosi2011a}{}
Piantadosi, S., Tily, H., \& Gibson, E. (2011). The communicative
function of ambiguity in language. \emph{Cognition}, \emph{122},
280--291.

\hypertarget{ref-RegierKempKay2015a}{}
Regier, T., Kemp, C., \& Kay, P. (2015). Word meanings across languages
support efficient communication. In B. M. \& W. O'Grady (Ed.), \emph{The
ahndbook of language emergence (pp. 237-263)}. Hoboken, NJ:
Wiley-Blackwell.

\hypertarget{ref-SmithTamarizKirby2013a}{}
Smith, K., Tamariz, M., \& Kirby, S. (2013). Linguistic structure is an
evolutionary trade-off between simplicity and expressivity. In
\emph{Proceedings of the 36th annual conference of the cognitive science
society}.

\hypertarget{ref-WasowPerforsBeaver2005a}{}
Wasow, T., Perfors, A., \& Beaver, D. (2005). The puzzle of ambiguity.
In CSLI (Ed.), \emph{Morphology and the web of grammar: Essays in memory
of steven g lapointe}. Stanford, CA: McGraw-Hill.

\hypertarget{ref-Zipf1949a}{}
Zipf, G. (1949). \emph{Human behavior and the principle of least
effort}. New York, NY: Prentice-Hall.

\bibliographystyle{apacite}


\end{document}
